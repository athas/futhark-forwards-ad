\documentclass[a4paper]{report}
\usepackage{mathpartir}
\usepackage{mathpazo}
\usepackage{amssymb,amsthm,amsmath,amsfonts}
\usepackage{graphicx,color}
\usepackage{booktabs}
\usepackage{listings}
\usepackage{hyperref}
\usepackage[shortlabels]{enumitem}
\usepackage[margin=1.0in, marginparwidth=2in]{geometry}

% Define Language
\lstdefinelanguage{futhark}
{
  % list of keywords
  morekeywords={
    do,
    else,
    for,
    fun,
    if,
    in,
    include,
    let,
    loop,
    struct,
    then,
    type,
    val,
    while,
    with,
    module,
    where,
  },
  sensitive=true, % keywords are not case-sensitive
  morecomment=[l]{--}, % l is for line comment
  morecomment=[s]{\{-}{-\}}, % s is for start and end delimiter
%  otherkeywords={>,<,=,<=,>=,!,*,/,-,+,|,&,||,&&,==,=>},
  morestring=[b]" % defines that strings are enclosed in double quotes
}

\lstdefinelanguage{corefuthark}
{
  % list of keywords
  morekeywords={
    do,
    else,
    for,
    fun,
    if,
    in,
    include,
    let,
    loop,
    struct,
    then,
    type,
    val,
    while,
    with,
    module,
    where,
  },
  sensitive=true, % keywords are not case-sensitive
  literate={\\}{\fn}{1} {->}{$\rightarrow$}{1} {<-}{$\leftarrow$}{1},
  moredelim=**[is][\color{red}]{@}{@},
  morecomment=[l]{--}, % l is for line comment
  morecomment=[s]{\{-}{-\}}, % s is for start and end delimiter
%  otherkeywords={>,<,=,<=,>=,!,*,/,-,+,|,&,||,&&,==,=>},
  morestring=[b]" % defines that strings are enclosed in double quotes
}

% Define Colors
\usepackage{xcolor}
\definecolor{eclipseBlue}{RGB}{42,0.0,255}
\definecolor{eclipseGreen}{RGB}{63,127,95}
\definecolor{eclipsePurple}{RGB}{127,0,85}

\newcommand{\fop}[1]{\mbox{\ttfamily\color{eclipseBlue}#1}}
\newcommand{\fw}[1]{\mbox{\ttfamily\bfseries\color{eclipsePurple}#1}}

% Set Language
\lstset{
  language={futhark},
  basicstyle=\small\ttfamily, % Global Code Style
  extendedchars=true, % Allows 256 instead of 128 ASCII characters
  tabsize=2, % number of spaces indented when discovering a tab
  columns=fixed, % make all characters equal width
  keepspaces=true, % does not ignore spaces to fit width, convert tabs to spaces
  showstringspaces=false, % lets spaces in strings appear as real spaces
  numbers=none, % do not show line numbers at the left
  numberstyle=\footnotesize\ttfamily, % style of the line numbers
  commentstyle=\itshape\color{eclipseGreen}, % style of comments
  keywordstyle=\bfseries, % style of keywords
  stringstyle=\color{eclipseBlue}, % style of strings
  emph=[1] {
    false,
    filter,
    iota,
    map,
    map2,
    map4,
    partition,
    rearrange,
    reduce,
    reduce_comm,
    redomap,
    scanomap,
    replicate,
    reshape,
    rotate,
    shape,
    scan,
    split,
    true,
    unzip,
    scatter,
    zip,
    stream_seq,
    stream_red,
    stream_map,
    stream_par,
    size,
    manifest,
    local,
    kernel,
    stream_group,
  },
  emphstyle=\ttfamily\bfseries,
  moredelim=**[is][\color{red}]{@}{@},
  aboveskip=-0.1\baselineskip,
  belowskip=\baselineskip,
}

\newcommand{\Dom}{{\rm Dom}}
\newcommand{\ov}[1]{\overline{#1}}
\newcommand{\nseq}[2]{\overline{#1}^{(#2)}}
\newcommand{\seq}[1]{\overline{#1}}
\newcommand{\LR}[1]{\langle #1\rangle}
\newcommand{\hsp}{\hspace{5mm}}
\newcommand{\kt}[1]{\textsf{#1}}
\newcommand{\kw}[1]{\mbox{\texttt{\bf #1}}}
\newcommand{\id}[1]{\mbox{\it{#1}}}
\newcommand{\M}[2]{\LR{#1\in #2}}
\newcommand{\Mv}[2]{\LR{\seq{#1}\in\seq{#2}}}
\newcommand{\Mvv}[4]{\LR{\seq{#1}\,\seq{#2}\in\seq{#3}\,\seq{#4}}}
\newcommand{\Do}{\kw{do}}
\newcommand{\For}{\kw{for}}
\newcommand{\Map}{\kw{map}}
\newcommand{\fn}{\ensuremath{\lambda}}
\newcommand{\Fn}[3]{\fn#2:~#1~\rightarrow #3}
\newcommand{\FnU}[2]{\fn#1~\rightarrow #2}
\newcommand{\Reduce}{\kw{reduce}}
\newcommand{\Reshape}{\kw{reshape}}
\newcommand{\Redomap}{\kw{redomap}}
\newcommand{\Scanomap}{\kw{scanomap}}
\newcommand{\Scan}{\kw{scan}}
\newcommand{\Transpose}{\kw{transpose}}
\newcommand{\Let}[3]{\kw{let}~#1~\mbox{\texttt{=}}~#2~\kw{in}~#3}
\newcommand{\Lett}[3]{\!\begin{array}[t]{l}\kw{let}~#1~\mbox{\texttt{=}}~#2 \\\kw{in}~#3 \end{array}}
\newcommand{\If}[3]{\kw{if}~#1~\kw{then}~#2~\kw{else}~#3}
\newcommand{\Iff}[5]{\begin{array}[t]{l}\kw{if}~#1~\kw{then}~ #2\\\kw{else}~\kw{if}~#3~\kw{then} ~#4 \\\kw{else}~#5\end{array}}
\newcommand{\Loop}[5]{\kw{loop}~#1~\texttt{=}~#2~\kw{for}~#3<#4~\kw{do}~#5}
\newcommand{\Loopp}[5]{\begin{array}[t]{l}\kw{loop}~#1~\texttt{=}~#2~\kw{for}~#3<#4~\kw{do}\\\hsp #5\end{array}}
\newcommand{\vd}{\vdash}
\newcommand{\Rearrange}{\kw{rearrange}}
\newcommand{\Replicate}{\kw{replicate}}
\newcommand{\Par}[1]{\mathtt{(}#1\mathtt{)}}
\newcommand{\SqPar}[1]{\mathtt{[}#1\mathtt{]}}
\newcommand{\Set}[1]{\{#1\}}
\newcommand{\StreamMap}{\kw{stream\_map}}
\newcommand{\StreamRed}{\kw{stream\_red}}
\newcommand{\StreamPar}{\kw{stream\_par}}
\newcommand{\StreamSeq}{\kw{stream\_seq}}
\newcommand{\StreamGroup}{\kw{stream\_group}}
\newcommand{\Segmap}{\kw{segmap}}
\newcommand{\Segred}{\kw{segred}}
\newcommand{\Segscan}{\kw{segscan}}
\newcommand{\sembox}[1]{\hfill \normalfont \mbox{\fbox{\(#1\)}}}
\newcommand{\sempart}[2]{\textrm{\textit{#1 \sembox{#2}}}}

\newcommand{\fract}[3]{\vspace{2mm}\mbox{$\frac{\begin{array}{c} #2 \end{array}}{\begin{array}{c} #3 \end{array}}$}~[\mbox{\textsc{#1}}]}
\newcommand{\onepart}[1]{\noindent\hfill#1\hfill\mbox{~}}
\newcommand{\twopart}[2]{\noindent\hfill#1\hfill#2\hfill\mbox{~}}
\newcommand{\threepart}[3]{\noindent\hfill#1\hfill#2\hfill#3\hfill\mbox{~}}

%%% Local Variables:
%%% mode: latex
%%% TeX-master: "icfp18"
%%% End:


%\newcommand{\ttb}[1]{\texttt{\bf #1}}
%\newcommand{\Loop}[5]{\ttb{loop}~#1=#2~\ttb{for}~#3<#4~\ttb{do}~#5}
\newcommand{\ctx}[2]{#1 ~{\parallel}~ #2}
\newcommand{\ctxa}[3]{#1 \vdash #2 ~{\parallel}~ #3}
\newcommand{\ctxb}[2]{#1 \vdash #2}
\newcommand{\LoopR}[5]{\kw{loop}~#1~\texttt{=}~#2~\kw{for}~#3=#4~\kw{to}~0~\kw{do}~#5}
\newcommand{\update}[2]{#1\mathrel{+}=#2}
\newcommand\doubleplus{+\kern-1.3ex+\kern0.8ex}

\begin{document}
\section*{Terms}
(Stolen from the Incremental Flattening paper\footnote{\url{https://futhark-lang.org/publications/ppopp19.pdf}} :) )
\[
\begin{array}{lcl}
  \id{bop} & ::= & \kw{+} ~~|~~ \kw{-} ~~|~~ \kw{*} ~~|~~ \kw{/} ~~|~~ \kw{<} ~~|~~ \cdots \\
  \id{op} & ::= & \Transpose ~~|~~ \Rearrange~\Par{d,\cdots{},d} ~~|~~ \Replicate \\
  \id{soac} & ::= & \Map ~~|~~ \Reduce ~~|~~ \Scan ~~|~~ \Redomap ~~|~~ \Scanomap\\
  e & ::= & x ~~|~~ d ~~|~~ b ~~|~~ \Par{e,\cdots{},e} ~~|~~ e\SqPar{e} ~~|~~ e~\id{bop}~e
      ~~|~~ \id{op}~e~\cdots{}~e ~~|~~ {\mathbf 0} \\
  & |   & \Loop{x_1~\cdots{}~x_n}{e}{y}{e}{e} \\
    & |   & \LoopR{x_1~\cdots{}~x_n}{e}{y}{e}{e} \\
    & |   & \Let{x_1~\cdots{}~x_n}{e}{e} ~~|~~ \If{e}{e}{e} \\
    & |   & \id{soac} ~f~ e~\cdots{}~e \\
  f & ::= & \FnU{x_1~\cdots{}~x_n}{e} ~~|~~ ~~\id{soac}~f~e~\cdots{}~e ~~|~~ e~\id{bop} ~~|~~ \id{bop}~e
\end{array}
\]
The ${\mathbf 0}$ expression is an array of zeros of arbitrary (and polymorphic!) shape.

\section*{Reverse-mode Rules}
We define \emph{tape maps} ($\parallel \Omega$) and \emph{adjoint contexts} $(\Lambda \vdash)$ as

\[
\begin{array}{lcl}
  \Omega ::= \varepsilon ~~|~~ \Omega, (x \mapsto x_{s}) \\
  \Lambda ::= \varepsilon ~~|~~ \Lambda, (x \mapsto \hat{x})
\end{array}
\]
The union of two maps prefers the right map in the instance of key conflicts:
\begin{align*}
  (\Lambda_1 \cup \Lambda_2)[x] = \begin{cases}
                                    \Lambda_2[x] & \text{if $(x \mapsto \hat{x}) \in \Lambda_2$} \\
                                   \Lambda_1[x] & \text{otherwise}
                                  \end{cases}
\end{align*}
Mappings of lists of variables is sugar for a list of mappings:
\begin{align*}
  x_1 x_2 \cdots x_n \mapsto \hat{x}_1 \hat{x}_2 \cdots \hat{x_n} = \epsilon, (x_1 \mapsto \hat{x}_1), (x_2 \mapsto \hat{x}_2), \dots, (x_n \mapsto \hat{x}_n)
\end{align*}
$dom$ returns all keys of a map, i.e.,
\[
dom\left(\epsilon, (x_1 \mapsto \hat{x}_1), (x_2 \mapsto \hat{x}_2)\right)= \{x_1, x_2\}
\]
$im$ returns all elements:
\[
im\left(\epsilon, (x_1 \mapsto \hat{x}_1), (x_2 \mapsto \hat{x}_2)\right)= \{\hat{x}_1, \hat{x}_2\}
\]
We're sloppy and overload the notation somewhat, so expressions like
\[
\Let{dom\left(\epsilon, (x_1 \mapsto \hat{x}_1), (x_2 \mapsto \hat{x}_2)\right)}{e_1}{e_2}
\]
are to be understood as
\[
\Let{x_1 ~ x_2}{e_1}{e_2}
\]
or
\[
\Let{(x_1, x_2)}{e_1}{e_2}
\]
depending on the context. The difference of two maps is defined as
\[
\Lambda_2 \setminus \Lambda_1 = \cup \{ x \mapsto \hat{x} \mid \Lambda_2[x] \neq \Lambda_1[x] \}
\]
\subsection*{Forward pass ($\Rightarrow_F$)} 
\begin{figure}[!h] 
\centering
\fbox{
\begin{mathpar}
\inferrule* [right=FwdLoop]
 {
 e = \Loop{\seq{x}}{e_0}{y}{e_n}{e_{body}} \\
  x_{s_0}~\text{fresh} \\
  x_{s_0} = \Replicate~e_n~{\mathbf 0}
}
{
 e \Rightarrow_{F}  \ctx{\Loop{(\seq{x}, x_{s})}{(e_0, x_{s_0})}{y}{e_{n}}{(e_{body}, x_s[y] = \seq{x})}}{(\seq{x} \mapsto x_{s})}
}

%\inferrule* [right=FwdIF]
%{
%  \If{e_p}{e_t}{e_f}
%  e_t \Rightarrow_F \ctx{e_t'}{\Omega_t}
%  e_f \Rightarrow_F \ctx{e_f'}{\Omega_f}
%}
%{
%  \If{e_p}{e_t}{e_f} \Rightarrow_F \ctx{\If{e_p}{e_t'}{e_f'}}{\Omega_t \cup \Omega_f}
%}
\end{mathpar}
}
\end{figure}
\newpage
\subsection*{Reverse pass ($\Lleftarrow$)}
\begin{figure}[!h] 
\centering
\fbox{
\begin{mathpar}
\inferrule* [right=RevLoop]
            {
              e_{body} = \Let{\seq{rs}}{e_{body}'}{\seq{rs}} \\
  e_{loop} = \Let{\seq{lres}}{\Loop{\seq{x}}{e_0}{y}{e_n}{e_{body}}}{\seq{lres}}\\
  e_{loop} \Rightarrow_F \ctx{e_{loop}'}{\Omega} \\
  \seq{fv} = FV(e_{body}) \setminus \seq{x} \\
  \seq{\hat{x}},~ \seq{\hat{fv}},~ \seq{\hat{fv}'},~\seq{\hat{fv}''},~\seq{\hat{rs}},~\seq{\hat{rs}'} ~fresh \\
  \seq{reset} = \Map~(\fn\_.~\mathbf 0)~\seq{\hat{x}} \\
  \Lambda_1' = \Lambda_1,~\seq{x} \mapsto \seq{\hat{x}},~\seq{fv} \mapsto \seq{\hat{fv}},~\seq{rs} \mapsto \seq{\hat{rs}} \\
  \hat{e}_{body} = \Let{\seq{\hat{z}}}{\hat{e}_{body}'}{\seq{\hat{z}}} \\
  (\ctxb{\Lambda_1'}{e_{body}}) \Lleftarrow (\ctxb{\Lambda_2}{\hat{e}_{body}}) \\
  \Lambda_{2,fv} = \{ v \mapsto \hat{v} \mid v \in \seq{fv}, (v \mapsto \hat{v}) \in \Lambda_{2} \setminus \Lambda_1'\} \\
  \Lambda_{2,rs} = \{ r \mapsto \hat{r} \mid r \in \seq{rs}, (r \mapsto \hat{r}) \in \Lambda_{2}\} \\
  \hat{e}_{body}'' = \Let{\seq{rs}}{\Omega[y]}{(\Let{\seq{\hat{z}'}}{\hat{e}_{body}'}{(\seq{reset}, im(\Lambda_{2,rs}), im(\Lambda_{2,fv})}))} \\
  \widehat{init} = (\seq{reset},\Lambda_1[\seq{lres}], \Lambda_1[dom(\Lambda_{2,fv})]) \\
  \hat{e}_{loop} = {\LoopR{(\seq{\hat{x}},\seq{\hat{rs}}, \seq{\hat{fv}})}{\widehat{init}}{y}{e_n - 1}{\hat{e}_{body}''}}\\
  \Lambda_3 = \Lambda_1 \cup \left(dom(\Lambda_{2,fv}) \mapsto \seq{\hat{fv}''}\right)
}
            {
                    \ctxb{\Lambda_1}{e_{loop}}
                    \Lleftarrow \left(\ctxb{\Lambda_3}{\Let{(\_~,\seq{\hat{rs}'}, \seq{\hat{fv}'})}{\hat{e}_{loop}}{(\Let{\seq{\hat{fv}''}}{(\Map~(+)~\seq{\hat{fv}'}~\seq{\hat{rs}'})}{\seq{\hat{fv}''}}})}\right)\\
            }

\inferrule* [right=RevIF]
            {
  \ctxb{\Lambda}{e_t} \Lleftarrow \ctxb{\Lambda_t}{\Let{\seq{\hat{fv}}_t}{\hat{e}_t}}{\seq{\hat{fv}}_t}\\
  \ctxb{\Lambda}{e_f} \Lleftarrow \ctxb{\Lambda_f}{\Let{\seq{\hat{fv}}_f}{\hat{e}_f}}{\seq{\hat{fv}}_f}\\
  \Lambda_{\Delta_t} = \Lambda \setminus \Lambda_t \\
  \Lambda_{\Delta_f} = \Lambda \setminus \Lambda_f \\
  \seq{\hat{fv}}~ fresh \\
  %\Lambda_{\cup} = \Lambda_{\Delta_t} \cup \Lambda_{\Delta_f} \\
  %\Lambda_t' = \Lambda_{\cup}\left[dom(\Lambda_{\Delta_f}) \mapsto \Lambda\left[dom(\Lambda_{\Delta f})\right]\right]\\
  %\Lambda_f' = \Lambda_{\cup}\left[dom(\Lambda_{\Delta_t}) \mapsto \Lambda\left[dom(\Lambda_{\Delta t})\right]\right]\\
  \hat{e}_t' = \Let{\seq{\hat{fv}}}{sort(\seq{\hat{fv}}_t ~ \doubleplus ~ im(\Lambda_{\Delta_f} - \Lambda_{\Delta_t}))}{(\Let{\seq{\hat{fv}}_t}{\hat{e}_t}{\seq{\hat{fv}}_t})} \\
  \hat{e}_f' = \Let{\seq{\hat{fv}}}{sort(\seq{\hat{fv}}_f ~ \doubleplus ~ im(\Lambda_{\Delta_t} - \Lambda_{\Delta_f}))}{(\Let{\seq{\hat{fv}}_f}{\hat{e}_f}{\seq{\hat{fv}}_f})} \\
  \Lambda' = \Lambda, ~(dom\Lambda_{\Delta_t} \cup (\Lambda_{\Delta_f}) \mapsto \seq{\hat{fv}})
}
{
  \ctxb{\Lambda}{\If{e_p}{e_t}{e_f}}\Lleftarrow \ctxb{\Lambda'}{\If{e_p}{\hat{e}_t'}{\hat{e}_f'}}
}
\end{mathpar}
}
\end{figure}

\end{document}
